\chapter{Instructions for Helium administrator}
Here are some tasks you need to do to set up Helium
to work before the tournament.

\section{Database maintenance}
\subsection{Clearing the database}
You should first clear out the old exams, verdicts, etc.\ from the database
before another run of the tournament.
If you want to save a copy of all the data,
you can do so using Django management:
\begin{center}
	\texttt{python manage.py dumpdata helium --format yaml}.
\end{center}
This will get you a YAML file with all Helium related data.
If you ever need it back you can then reload it with \texttt{loaddata}.

\subsection{Import exams from Babbage}
You need to populate the Helium tables with exams.

To do this, you should go to the problem database and for each contest
that you want to export, push the ``Export to Helium'' link.
This will copy all the problems and their answers into Helium,
and then open the admin interface
so that you can make any manual changes necessary
(for example partial marks are disabled by default
but you will need to add them into Guts estimation).
Remember to input colors corresponding to how the answer sheets will be tinted.

A safety feature is that exam names in Helium must be distinct.
So if you decide you want to re-import an exam,
you will have to delete the old one first.

\subsection{Import entities from registration}
In Babbage a command called ``Import Entities from Reg'' will do this for you.
It is careful, and will never import the same entity twice.

\section{Guts estimation problems}
First, ensure that the name of the Guts exam is ``Guts'',
or otherwise maul \texttt{babbage/views.py} to make this work.

Next, if there are any estimation problems the scoring functions
for these need to be added.
You can do this in the admin interface, under ``Guts Score Funcs''.
This should be entirely self-explanatory.

\section{Printing and checking scans}
You should make sure that a large number of answer sheets are printed,
and in different colors between tests to avoid errors.

Unfortunately, right now the scan regions (i.e.\ the cut-outs)
are hard-coded into \verb+scanimage.py+.
So before the tournament you should check the scanners being used
and verify that the cut-out regions are compatible with the scanner.

The file \verb+scanimage.py+ has a facility to make this easier.
If you run
\begin{center}
\verb+python /path/to/scanimage.py FILENAME.pdf+
\end{center}
then it will generate (in the current directory)
the images corresponding to the cut-outs.
Thus you can test this locally and make adjustments to the regions if needed.

\section{Test data}
\subsection{Location}
Some test data has been provided for you.
You can look for it in the following places:
\begin{itemize}
	\ii \verb+:/helium/static/batchX-samplescans.pdf+
	\ii \verb+:/helium/static/batchY-samplescans.pdf+
	\ii \verb+:/helium/fixtures/*+
\end{itemize}
The easiest thing to do is to load the two fixtures in the ``setup'' directory,
which will populate the exams and entities tables.
The sample scans correspond exactly to the entities in these fixtures.

You can also load fixtures in the \verb+:/helium/fixtures/scenarios+ directly
which are snapshots of the entire database at various points in time
in the test run
(after matching exams, after grading, after algorithmic scoring, etc.).
But note that the \verb+/media/+ folder is NOT included.

\emph{Warning}: the scenarios folder contains a superuser
with name and password ``evan''
as well as a staff user with name and password ``staff1''.
Consequently, be careful to NOT run these on production,
as we will then have some very insecure user accounts!

\subsection{Synopsis of test data}
Synopsis of test data:

\begin{itemize}
\ii There is a single scan-based exam, called ``GEO!'',
and there is a single non-scanned exam, called ``Mock IMO''.
The former is $10$ problems scored algorithmically.
The latter is $3$ problems worth $7$ points each.

\ii There are $16$ mathletes split into three teams of five,
corresponding to characters from \emph{L\'es Miserables},
\emph{Wicked}, and the \emph{Phantom of the Opera},
plus one unaffiliated individual named ``Evan Chen''.

\ii The two sample batches contain all their ``GEO!'' exams.
The answers to that exam corresponding exactly to the real answers
to the HMMT February 2016 Geometry test.

\ii Under the registration folder are some fixtures for testing
import to registration and so on.

\ii Under the problems folder are some fixtures for loading
some example data into the problem database.
\end{itemize}

