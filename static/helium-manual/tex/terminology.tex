\chapter{Terminology}

The grading system is itself known by the name \vocab{Helium}.
Each event will be referred to as a \vocab{contest}.

\section{Participants}
In each contest there are several participants.
The students are referred to as \vocab{mathletes}.
These students are often grouped into \vocab{teams}.
A mathlete may have no team; in this case we say they
are an \vocab{unaffiliated individual},
or sometimes just ``individual'' for short although this is confusing.

An \vocab{entity}\footnote{
	This abstraction is useful because an exam doesn't care so much who takes it.}
refers to someone,
either a mathlete or a team,
who takes an exam.

A \vocab{user} refers to a staff member of the contest
(not a student or coach).
A \vocab{superuser} refers to a higher-level administrator of the contest;
for example a head grader, or problem czar,
or tournament director, or software director.
To be more exact this is the set of users with administrative
privileges on the Django website.

\section{Exams and problems}
A contest consists of several \vocab{exams}\footnote{
	Helium deliberately avoids the word ``test'',
	since this may be confused with ``testing'' and so on},
each of which has several \vocab{problems},
which are numbered $1$, $2$, \dots, $n$ in each exam.
These exams are \vocab{taken} by entities,
who must attempt to solve the problems on the exam.

A problem has the following properties:
\begin{itemize}
	\ii The problem has a \vocab{weight},
	which is the number of points a contestant earns for solving it.

	\ii The problem may allow \vocab{partial} marks,
	or it may be an ``all-or-nothing'' problem.
\end{itemize}
Exams may be ``individual'' or ``team'' exams,
depending on which types of entities take them.

\section{Scans}
A \vocab{problem scribble} refers to the scan
of a contestant's answer to a problem.
An \vocab{exam scribble} refers to the scan
of a contestant's entire answer sheet for an exam.
You won't see these two terms much in the user interface,
but they are used internally in the source code just about everywhere.

A \vocab{paper} refers to the set of answers for
a contestant to any given exam,
regardless of whether the paper was fed into a scanner or not.

\section{Grading}
Suppose an entity $E$ attempts a problem $P$,
Helium then needs to give a \vocab{verdict} to the pair $(E,P)$,
which consists of a score assigned to the pair $(E,P)$.
This represents the judgment handed down by Helium to $E$
for this problem $P$.

To generate this verdict,
one or more users will input a score for the pair $(E,P)$.
Each such score by a grader is termed an \vocab{evidence}
contributing towards the verdict,
and we say the user \vocab{submits} this evidence.
The verdict is \vocab{valid} if there is a clear consensus what the final
score should be (by default this is at least a 3-to-1 majority,
but this may be adjusted per exam).
The verdict is \vocab{done} once it is valid and there is enough evidence.

There is one possible exception:
super-users may submit a evidence in \vocab{God mode}.
When this occurs, it overrides any and all evidence submitted by regular users.

\section{Algorithmic scoring}
At HMMT a complex algorithm is used for assigning
the weights to the individual tests.
Details of this algorithm are available at
\begin{quote}
	\url{https://www.hmmt.co/static/scoring-algorithm.pdf}
\end{quote}
A copy of this algorithm is also included in an appendix
at the end of this document.

Let $\mathcal A$ be the set of exams scored using this algorithm.
Each problem on an exam in $\mathcal A$
obtains a weight which is called a \vocab{beta value},
denoted $3 \le \beta \le 8$.
Moreover, each entity taking one or more exams in $\mathcal A$
is then assigned an \vocab{alpha value},
which represents their strength, and is denoted $\alpha \in [0,\infty)$.
These $\alpha$ and $\beta$ values are stored in database,
since they are quite hard to compute.

At HMMT the alpha values are used to aggregate individual scores
across all tests, and thus used to rank individuals.

