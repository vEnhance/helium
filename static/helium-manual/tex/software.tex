\chapter{Notes for software team}
This contains low-level technical notes on some of the non-obvious
moving parts behind the whole system.
This is intended to help you extend Helium in some way.

Most of the code I tried to comment or \verb+help_text+ well.
But if anything is still confusing, that's probably my fault.
Feel free to contact me at \texttt{chen.evan6@gmail.com}.

\section{Personal request}
I have a personal request before you start editing,
which is that my name remains intact in \texttt{LICENSE.txt},
this documentation, and other places (for example file headers).
You are free of course to add your own name
should you make substantial contributions.

The reason I ask this is because this system represents
well over $100$ volunteered hours of my free time
pooled into over $300$ commits.
I built the entire system from scratch with no assistance from anyone else.
Since I didn't collect any payment for this work,
the least I ask is that I am recognized for it.

Thanks!

\section{Moving parts}
When possible I tried to keep everything in Helium 
self-contained, not relying on even files outside the directory.

\subsection{Helium dependencies}
Here is a partial list of things externally that Helium relies on;
it may be incomplete.
All this is at time of writing;
other people move things around all the time so God knows
if we will even have \texttt{core/settings.py} in a few years.

\begin{itemize}
	\ii \texttt{requirements.txt} in the root directory
	has some Helium specific additions.
	In particular \texttt{wand} which does image processing in Python.
	The command
	\begin{center}
		\texttt{pip install -r requirements.txt}
	\end{center}
	should do the trick.

	\ii Image installation is done through the EB extension \texttt{01yum}.
	This installs \texttt{imagemagick}, \texttt{freetype}, \texttt{ghostscript},
	which is used for the image processing.

	\ii Helium needs to host its media on an external S3 server,
	because scan images need to be distributed.
	So currently I have an S3 bucket called \texttt{heliummedia}.
	The connection is in \texttt{core/settings.py},
	The S3 bucket itself lives inside AWS along with the entire HMMT application.
	
	\ii \texttt{core/settings.py} should also have some basic applications
	installed, such as
	\begin{itemize}
		\ii `django.contrib.admin',
		\ii `django.contrib.auth',
		\ii `django.contrib.sessions',
		\ii `django.contrib.messages',
		\ii `django.contrib.staticfiles',
		\ii `storages'.
	\end{itemize}
\end{itemize}

\subsection{Things depending on Helium}
\begin{itemize}
	\ii The Guts scoreboard reads its data from Helium.
	So if you make changes to Helium, check there too.
	
	\ii Registration import into Helium, as described before.
\end{itemize}

\subsection{Other nice things to know}
Some not terribly Helium-specific things, but you should still know:
\begin{itemize}
	\ii Logging configuration (not specific to Helium, but nice!)
	is done by the EB extension \texttt{00logging}
	and configured in \texttt{core/settings.py}.
	
	\ii Not Helium-specific, but the EB extension \texttt{05managepy}
	is responsible for running \texttt{collectstatic} and \texttt{migrate}.
\end{itemize}

\section{A wishlist}
Some possible to-do things to do:
\begin{itemize}
	\ii Image processing: auto align scan regions and cut-outs.
	\ii Auto import exams from HMMT Problems Database.
	\ii Proof grading (long shot).
\end{itemize}
