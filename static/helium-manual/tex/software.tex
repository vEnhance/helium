\chapter{Notes for software team}
This contains low-level technical notes on some of the non-obvious
moving parts behind the whole system.
This is intended to help you extend Helium in some way.

Most of the code I tried to comment or \verb+help_text+ well.
But if anything is still confusing, that's probably my fault.
Feel free to contact me at \texttt{chen.evan6@gmail.com}.

\section{Personal request}
I have a personal request before you start editing,
which is that my name remains intact in \texttt{LICENSE.txt},
this documentation, and other places (for example file headers).
You are free of course to add your own name
should you make substantial contributions.

The reason I ask this is because this system represents
well over $200$ volunteered hours of my free time
pooled into over $400$ commits.
I built the entire system from scratch with no assistance from anyone else.
Since I didn't collect any payment for this work,
the least I ask is that I am recognized for it.

Thanks!

\section{A Warning}
I found out the hard way that a typical tournament has maybe $N = 30000$ verdicts in it.
This means you should make sure your code does not do either of the following per call:
\begin{itemize}
	\ii Do not use algorithms that run in $O(N^2)$ time.
	\ii Do not make $O(N)$ queries to the database.
\end{itemize}
For queries grabbing the entire verdicts table (or large portions of it),
it is almost always better to use the \texttt{queryset.values} method.
For these large values of $N$, creating $N$ instances from the table is just too slow.

\section{Test data}
\subsection{Location}
Some test data has been provided for you.
You can look for it in the following places:
\begin{itemize}
	\ii \verb+:/helium/static/batchX-samplescans.pdf+
	\ii \verb+:/helium/static/batchY-samplescans.pdf+
	\ii \verb+:/helium/fixtures/*+
\end{itemize}
The easiest thing to do is to load the two fixtures in the ``setup'' directory,
which will populate the exams and entities tables.
The sample scans correspond exactly to the entities in these fixtures.

You can also load fixtures in the \verb+:/helium/fixtures/scenarios+ directly
which are snapshots of the entire database at various points in time
in the test run
(after matching exams, after grading, after algorithmic scoring, etc.).
But note that the \verb+/media/+ folder is NOT included.

\emph{Warning}: the scenarios folder contains a superuser
with name and password ``evan''
as well as a staff user with name and password ``staff1''.
Consequently, be careful to NOT run these on production,
as we will then have some very insecure user accounts!

\subsection{Synopsis of test data}
Synopsis of test data:

\begin{itemize}
\ii There is a single scan-based exam, called ``GEO!'',
and there is a single non-scanned exam, called ``Mock IMO''.
The former is $10$ problems scored algorithmically.
The latter is $3$ problems worth $7$ points each.

\ii There are $16$ mathletes split into three teams of five,
corresponding to characters from \emph{L\'es Miserables},
\emph{Wicked}, and the \emph{Phantom of the Opera},
plus one unaffiliated individual named ``Evan Chen''.

\ii The two sample batches contain all their ``GEO!'' exams.
The answers to that exam corresponding exactly to the real answers
to the HMMT February 2016 Geometry test.

\ii Under the registration folder are some fixtures for testing
import to registration and so on.

\ii Under the problems folder are some fixtures for loading
some example data into the problem database.
\end{itemize}

\section{Moving parts}
When possible I tried to keep everything in Helium 
self-contained, not relying on even files outside the directory.

\subsection{Helium dependencies}
Here is a partial list of things externally that Helium relies on;
it may be incomplete.
All this is at time of writing;
other people move things around all the time so God knows
if we will even have \texttt{core/settings.py} in a few years.

\begin{itemize}
	\ii \texttt{requirements.txt} in the root directory
	has some Helium specific additions.
	In particular \texttt{wand} which does image processing in Python.
	The command
	\begin{center}
		\texttt{pip install -r requirements.txt}
	\end{center}
	should do the trick.

	\ii Image installation is done through the EB extension \texttt{01yum}.
	This installs \texttt{imagemagick}, \texttt{freetype}, \texttt{ghostscript},
	which is used for the image processing.

	\ii Helium needs to host its media on an external S3 server,
	because scan images need to be distributed.
	So currently I have an S3 bucket called \texttt{heliummedia}.
	The connection is in \texttt{core/settings.py},
	The S3 bucket itself lives inside AWS along with the entire HMMT application.
	
	\ii \texttt{core/settings.py} should also have some basic applications
	installed, such as
	\begin{itemize}
		\ii `django.contrib.admin',
		\ii `django.contrib.auth',
		\ii `django.contrib.sessions',
		\ii `django.contrib.messages',
		\ii `django.contrib.staticfiles',
		\ii `storages'.
	\end{itemize}
\end{itemize}

\subsection{Things depending on Helium}
\begin{itemize}
	\ii The Guts scoreboard reads its data from Helium.
	So if you make changes to Helium, check there too.
	
	\ii Registration import into Helium, as described before.
\end{itemize}

\subsection{Other nice things to know}
Some not terribly Helium-specific things, but you should still know:
\begin{itemize}
	\ii Logging configuration (not specific to Helium, but nice!)
	is done by the EB extension \texttt{00logging}
	and configured in \texttt{core/settings.py}.
	
	\ii Not Helium-specific, but the EB extension \texttt{05managepy}
	is responsible for running \texttt{collectstatic} and \texttt{migrate}.
\end{itemize}

\section{Git subtree}
There is an upstream URL which publishes Helium-specific code.
This way other math tournaments can use the same code themselves.

This upstream URL is maintained by a subtree via the command
\begin{center}
	\texttt{git subtree split --prefix=helium -b helium}
\end{center}
to create a branch \texttt{helium} containing only code from this folder.
Thereafter, we push this branch into the Git helium repository
(to the master branch on Git: note that the branch names differ).

The \texttt{README.txt} files provides a brief documentation
on how to get set up with Helium in a different Django project.

\section{A wishlist}
Some possible to-do things:
\begin{itemize}
	\ii Image processing: auto align scan regions and cut-outs.
	\ii Proof grading (long shot).
\end{itemize}
