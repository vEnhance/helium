\chapter{HMMT Algorithmic Scoring Details}
\section{Synopsis}
\subsection{Input}
Each exam contains at least one \textbf{problem}.
The exams are taken by several \textbf{contestants}.

This algorithm takes several \textbf{observations}:
i.e.\ it takes as inputs pairs $(c,p)$ of contestants and problems,
and for each such pair is told whether or not $c$ \textbf{solved} $p$.
During the actual tournament, every problem is attempted by every contestant;
however, the algorithm will not use this assumption.

\subsection{Output}
Based on this, the algorithm outputs:
\begin{itemize}
	\ii An \textbf{strength} $\alpha_c \in \RR_{\ge 0}$ for each competitor $c$, and
	\ii A \textbf{weight} $\beta_p \in [3,8]$ for each problem $p$,
	which represents its difficulty.
\end{itemize}
Note that the strength is \emph{not} the same as the score of a contestant.
In particular, as described below,
each contestant is given a single strength value across all exams.

\subsection{Usage in contest}
The calculation at the actual HMMT is done in one pass.
For example, at HMMT February 2016, there were three exams with $10$ problems
each and $700$ contestants.
Thus there were $700 \cdot 10 \cdot 3 = 21000$ observations as input.
Note that the algorithm is applied only once!
For example the observations on the Algebra test
affect the weights of the Geometry test.

For each particular 10-problem \textbf{exam},
the \textbf{score} of a contestant on that exam is the sum of the 
difficulties $\beta_p$ for each problem that they solve.
The score of a contestant is used to determine their rankings
within each individual test.
Scores across different exams are not comparable.

For individual sweepstakes, the strength $\alpha_c$ is used instead
(in order to ensure that no particular exam dominates
in determining the aggregate individual score).
Contestants are ranked based on the value of $\alpha_c$ assigned to them.
When determining team sweepstakes,
the individual contribution (out of $800$)
is proportional to the sum of $\alpha_c$ across the team members $c$,
scaled so that the strongest teams earns the maximal $800$ points.

\section{Design}
\subsection{Advantages}
The scoring algorithm is designed to meet the following criterion:
\begin{itemize}
	\ii The system provides a careful way to
	\textbf{compare scores across tests},
	leading to less noise in the aggregate individual
	and team rankings.

	\ii The system \textbf{factors all possible inputs},
	rather than for example merely the top 10 contestants or otherwise,
	as occurred in previous HMMT tournaments.

	\ii The system is \textbf{resistant to preconceived difficulty}.
	Originally, problem czars were forced to estimate the difficulty
	of every problem by assigning it a weight;
	this leads to possible noise in the results.
	The algorithmic system determines the difficulty based on the actual performance.

	\ii The system is \textbf{hard to exploit},
	in part because it takes inputs from all contestants,
	and in part because it is so complicated
	that a contestant is probably better off thinking about
	the exam problems.
\end{itemize}

\subsection{Qualitative criteria}
The following additional criteria are satisfied:
\begin{itemize}
	\ii The strength of each contestant is nonnegative,
	\ii The strength of a contestant is unbounded,
	but not infinite even if they solve every problem.
	\ii The strength of a contestant is zero
	if they solve no problems.
	\ii Problem weights lie in the interval $(3,8)$.
	\ii The distribution functions are smooth.
\end{itemize}

\subsection{Choice of priors}
In light of this, we select the following parameters,
which describe how the $\alpha_c$ and $\beta_p$ should be interpreted.
\begin{itemize}
	\ii The strength $\alpha = \alpha_c$ of each contestant
	$c$ is \emph{a priori} distributed in $\RR_{\ge 0}$
	according to
	\begin{equation}
		\mathbf P(\alpha) = \exp(-\alpha).
		\label{eq:cprior}
	\end{equation}
	\ii The weight (difficulty) $\beta = \beta_p$ of each problem $p$ is
	\emph{a priori} distributed in $[3,8]$ according to 
	\begin{equation}
		\mathbf P(\beta) = \exp\left( \frac{5}{(\beta-3)(8-\beta)} \right).
		\label{eq:wprior}
	\end{equation}
\end{itemize}
Now we relate the strength $\alpha = \alpha_c$ of contestant $c$
to the difficulty $\beta = \beta_p$ of problem $p$ by
postulating that the probability that $c$ solves $p$ is
\begin{equation}
	\mathbf P
	\left( \text{$c$ solves $p$} 
	\mid \alpha_c = \alpha \text{ and } \beta_p = \beta \right)
	= \frac{\exp(-\beta/\alpha)}{1 + \exp(-\beta/\alpha)}.
	\label{eq:prob}
\end{equation}
Assuming \eqref{eq:cprior}, \eqref{eq:wprior}, \eqref{eq:prob}
and holding fixed the set of observations of the tournament:
\begin{quote}
\itshape
The algorithm outputs the unique set of $\alpha_c$ and $\beta_p$
for which the outcome of the tournament was most likely.
\end{quote}
The rest of the document describes how to actually compute
the maximum.

\section{Algorithm Description}
The algorithm is a slightly modified version of the Rasch model.

\subsection{Pseudocode}
Let us for brevity denote the function in \eqref{eq:prob} by 
\[ w(\alpha, \beta) \defeq \frac{\exp(-\beta/\alpha)}{1 + \exp(-\beta/\alpha)}. \]
Note that $w(\alpha, \beta) \in (0,1)$,

Now consider the following systems of equations,
whose origin we explain below.
First, for every $\alpha = \alpha_c$ we have the equation
\begin{equation}
	0 = -1 + \sum_{\text{$p$ solved by $c$}} \beta_p\alpha^{-2}
		- \sum_{\text{$p$ took by $c$}} \beta_p \alpha^{-2} w(\alpha, \beta_p).
	\label{eq:alpha}
\end{equation}
Moreover, for every $\beta = \beta_p$ we have the equation
\begin{equation}
	0 = \frac{1}{(\beta-3)^2} - \frac{1}{(8-\beta)^2}
		+ \sum_{\text{$c$ taken by $p$}} \alpha_c^{-1} w(\alpha_c, \beta)
		- \sum_{\text{$c$ solved by $p$}} \alpha_c\inv.
	\label{eq:beta}
\end{equation}
The algorithm binary searches for a set of $\vec\alpha$ and $\vec\beta$
which simultaneously satisfy \eqref{eq:alpha} and \eqref{eq:beta}.
It is proved below that this set of parameters is unique;
this is the output of the algorithm.


\subsection{Derivation}
We show the derivation of equations \eqref{eq:alpha} and \eqref{eq:beta}.

Fix a set of observations for the tournament.
Now consider a choice of $\vec\alpha$ and $\vec\beta$ is known.
Then the probability of the observations is distributed according to
\begin{equation}
	F( \vec\alpha, \vec\beta )
	= \prod_{c} \mathbf P(\alpha_c)
	\prod_{p} \mathbf P(\beta_p)
	\prod_{\text{$c$ solved $p$}} w(\alpha_c, \beta_p)
	\prod_{\text{$c$ missed $p$}} (1-w(\alpha_c, \beta_p))
	\label{eq:F}
\end{equation}
assuming the events are independent.

We now claim that:
\begin{proposition}
	If a choice of $\vec\alpha$ and $\vec\beta$ optimizes $F$,
	then it is a solution to \eqref{eq:alpha} and \eqref{eq:beta}.
	In other words, the outputted $\vec\alpha$ and $\vec\beta$
	are those for which the observed outcome was most probable.
\end{proposition}

\begin{proof}
	Note that
	\[
		\prod_{\text{$c$ solved $p$}} w(\alpha_c, \beta_p)
		\prod_{\text{$c$ missed $p$}} (1-w(\alpha_c, \beta_p))
		=
		\prod_{\text{$c$ solved $p$}} e^{-\beta_p/\alpha_c}
		\prod_{\text{$c$ took $p$}} \frac{1}{1+e^{-\beta_p/\alpha_c}}.
	\]
	From this, and plugging in \eqref{eq:cprior} and \eqref{eq:wprior}, we get
	\[
		F(\vec\alpha, \vec\beta)
		= \prod_{c} e^{-\alpha_c}
		\prod_{p} e^{-\frac{5}{(8-\beta_p)(\beta_p-3)}}
		\prod_{\text{$c$ solved $p$}} e^{-\beta_p/\alpha_c}
		\prod_{\text{$c$ took $p$}} \frac{1}{1+e^{-\beta_p/\alpha_c}}.
	\]
	It is equivalent to maximize $\log F$, which is
	\[
		\log F
		= -\sum_{c} \alpha_c
		- \sum_{p} \frac{5}{(8-\beta_p)(\beta_p-3)}
		- \sum_{\text{$c$ solved $p$}} \frac{\beta_p}{\alpha_c}
		+ \sum_{\text{$c$ took $p$}} \log\frac{1}{1+e^{-\beta_p/\alpha_c}}.
	\]
	Then right-hand sides of \eqref{eq:alpha} and \eqref{eq:beta}
	are the \emph{partial derivatives} of $\log F$ with respect
	to $\alpha_c$ and $\beta_p$ for each $c$ and $p$.
	The partial derivatives equal zero at any local maximum of $\log F$,
	as desired.
\end{proof}

In fact, one can actually check that the right-hand sides of 
\eqref{eq:alpha} and \eqref{eq:beta}
are monotonic in $\alpha$ and $\beta$ respectively.
Consequently, $F$ is convex.
This implies:
\begin{proposition}
	The function $\log F$ has a unique point at which all partial
	derivatives vanish, and that point is a maximum for $F$.
\end{proposition}

This also implies that one can numerically solve the equations
by the following binary search procedure.
We let $\vec\alpha_0$ be an arbitrary point.
Then we let $\vec\beta_n$ be the solution to \eqref{eq:beta} given $\vec\alpha_{n-1}$.
Notice that one can actually solve for each component $\beta_p$
using binary search separately,
as each equation of the form \eqref{eq:beta} involves only a single $\beta_p$.
Similarly, we can let $\vec\alpha_n$ be the solution to \eqref{eq:alpha}
given $\vec\beta_{n-1}$ in the same fashion.
The pairs $(\vec\alpha_n, \vec\beta_n)$ will converge to the maximum
since we are looking at partial derivatives of a concave function,
so we now have an iterative procedure for computing the output.
